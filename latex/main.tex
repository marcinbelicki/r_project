\newcommand*{\includesDirectory}{includes}
\newcommand*{\settingsDirectory}{\includesDirectory/settings}

\documentclass[a4paper,12pt,oneside]{book}
\usepackage{geometry}

\usepackage{setspace}

\geometry{
    left=  20mm,
    right= 20mm,
    top=   20mm,
    bottom=20mm
}

\setlength{\parindent}{0mm}

\usepackage[T1]{fontenc}
\usepackage[polish]{babel}
\usepackage{lmodern}
\usepackage{titletoc}
\usepackage{hyperref}
\usepackage[all]{hypcap}
\setcounter{tocdepth}{1}

\renewcommand\thechapter{Rozdział \Roman{chapter}}
\renewcommand\thesection{\arabic{section}.}
\renewcommand\thesubsection{\arabic{section}.\arabic{subsection}}
\contentsmargin{0.5cm}
\newlength\chapterlength
\settowidth\chapterlength{\hspace{2.5cm}}

\titlecontents{chapter}
    [\chapterlength] %5.3
    {\vspace{0.2cm}}
    {\contentslabel[\thecontentslabel]{\chapterlength}}%\thecontentslabel
    {\hspace*{-\chapterlength}}% unnumbered chapters
    {\titlerule*[1cm]{.}\contentspage}[\vspace{0.2cm}]%

\titlecontents{section}
    [\chapterlength] %5.3
    {\small}
    {\contentslabel[\thecontentslabel]{0.5cm}}
    {}
    {\titlerule*[0.5cm]{.}\contentspage}[]




\usepackage{titlesec}


\titleformat{\chapter}[block]
  {\normalfont}
  {\Huge{\thechapter\vspace{-13pt}\\}}
  {0pt}
  {\Large}
\titlespacing*{\chapter}{0pt}{0pt}{10pt}

\newcommand{\chapterwithout}[1]{\chapter*{#1} 
\addcontentsline{toc}{chapter}{#1}  
}

\titleformat{\section}
  {\normalfont}
  {\makebox[0.5cm][l]{\thesection}}
  {10pt}
  {}
\titlespacing*{\section}{0pt}{10pt}{10pt}

\titleformat{\subsection}[block]
  {\normalfont}
  {\hspace{1cm}\makebox[0.5cm][l]{\thesubsection}}{10pt}{}
\titlespacing*{\subsection}{0pt}{5pt}{5pt}

\renewcommand{\labelenumi}{$\arabic{enumi}^\circ$}


\begin{document}
	\thispagestyle{empty}
	\vspace*{\fill}
			\begin{center}
				\Huge{
					\textit{\texttt{PROJEKT JĘZYK R}}
				}
			\end{center}
	\vspace*{\fill}
			\begin{tabular}{ll}
				\texttt{zbiór danych:}  & \texttt{Video Game Sales \cite{source}}\\
				\texttt{autor:}         & \texttt{Marcin Belicki}\\
				\texttt{numer indeksu:} & \texttt{273417}
			\end{tabular}
	\newpage
	\setlength{\parindent}{9mm}
	
	\section{Opis zbioru danych}
		Zbiór danych zawiera listę gier ze sprzedażą przekraczającą 100 000 kopii. Dane zostały pozyskane ze strony \href{https://www.vgchartz.com/}{vgchartz.com}.\\
		Opisy poszczególnych pól:
		
	

	
	\section{Ilustracja graficzna zmiennej głównej}
	
	\section{Obliczenie podstawowych statystyk opisowych zmiennej głównej}
	
	\section{Dobór zmiennych (przynajmniej trzech), które mogą mieć wpływ na zmienną główną}
	
	\section{Graficzna prezentacja wybranych w poprzednim kroku zmiennych}
	
	\section{Charakterystyka powyższych zmiennych (statystyki opisowe lub rozkłady liczebności, w zależności od klasy zmiennych)}
	
	\section{Graficzna prezentacja zależności}
	
	\section{Wykonanie odpowiedniego testu statystycznego, który potwierdzi lub odrzuci hipotezę o zależności}

	\section{Wnioski}


	\begin{thebibliography}{1}
    \addcontentsline{toc}{chapter}{Bibliografia}
    \bibitem{source}
    \url{https://www.kaggle.com/datasets/gregorut/videogamesales}
    
    [dostęp: 13.06.2022 22:37 GMT+2]
\end{thebibliography}
\end{document}